\documentclass[11pt]{article}
\usepackage{amsmath,amssymb,amsthm}
\usepackage{algorithm}
\usepackage[noend]{algpseudocode} 

%---enable russian----

\usepackage[utf8]{inputenc}
\usepackage[russian]{babel}

% PROBABILITY SYMBOLS
\newcommand*\PROB\Pr 
\DeclareMathOperator*{\EXPECT}{\mathbb{E}}


% Sets, Rngs, ets 
\newcommand{\N}{{{\mathbb N}}}
\newcommand{\Z}{{{\mathbb Z}}}
\newcommand{\R}{{{\mathbb R}}}
\newcommand{\Zp}{\ints_p} % Integers modulo p
\newcommand{\Zq}{\ints_q} % Integers modulo q
\newcommand{\Zn}{\ints_N} % Integers modulo N

% Landau 
\newcommand{\bigO}{\mathcal{O}}
\newcommand*{\OLandau}{\bigO}
\newcommand*{\WLandau}{\Omega}
\newcommand*{\xOLandau}{\widetilde{\OLandau}}
\newcommand*{\xWLandau}{\widetilde{\WLandau}}
\newcommand*{\TLandau}{\Theta}
\newcommand*{\xTLandau}{\widetilde{\TLandau}}
\newcommand{\smallo}{o} %technically, an omicron
\newcommand{\softO}{\widetilde{\bigO}}
\newcommand{\wLandau}{\omega}
\newcommand{\negl}{\mathrm{negl}} 

% Misc
\newcommand{\eps}{\varepsilon}
\newcommand{\inprod}[1]{\left\langle #1 \right\rangle}

 
\newcommand{\handout}[5]{
  \noindent
  \begin{center}
  \framebox{
    \vbox{
      \hbox to 5.78in { {\bf Научно-исследовательская практика} \hfill #2 }
      \vspace{4mm}
      \hbox to 5.78in { {\Large \hfill #5  \hfill} }
      \vspace{2mm}
      \hbox to 5.78in { {\em #3 \hfill #4} }
    }
  }
  \end{center}
  \vspace*{4mm}
}

\newcommand{\lecture}[4]{\handout{#1}{#2}{#3}{Scribe: #4}{Быстрое умножение методом Тоома-Кука #1}}

\newtheorem{theorem}{Теорема}
\newtheorem{lemma}{Лемма}
\newtheorem{definition}{Определение}
\newtheorem{corollary}{Следствие}
\newtheorem{fact}{Факт}

% 1-inch margins
\topmargin 0pt
\advance \topmargin by -\headheight
\advance \topmargin by -\headsep
\textheight 8.9in
\oddsidemargin 0pt
\evensidemargin \oddsidemargin
\marginparwidth 0.5in
\textwidth 6.5in

\parindent 0in
\parskip 1.5ex

\begin{document}

\lecture{}{Лето 2020}{}{Иванов Артём Евгеньевич}

\section{Теория}

\textbf{ Алгоритм Тоома-Кука } — алгоритм, предназначенный для умножения длинных целых чисел. Даны 2 длинных числа $a$ и $b$, алгоритм разбивает их на $k$ маленьких чатей длины $l$ и проводит вычисления.  Toom-3 - это частный случай алгоритма Тоома-Кука, где k = 3. Toom-3 сокращает количество умножений с 9 до 5 и выполняется за $\mathcal{O}(n^{log(5)/log(3)})\approx\mathcal{O(}n^{1.46})$

\section{Алгоритм}

\begin{algorithm}[ph]
	\caption{Быстрое умножение методом Тоома-Кука (Toom-3)}
	\label{alg:AlgName}
	\textbf{Ввод:} целые n-разрядные  X и Y, базис разложения B  \\
	\textbf{Вывод:} $Z = X\cdot Y$
	
	\begin{algorithmic}
		
		\State 1.	 Разложить X и Y по базису B
			\State \ \ \ \ \ \	$X =  X_2B^2 +X_1B+X_0$ 
			\State \ \ \ \ \ \	$Y =  Y_2B^2 +Y_1B+Y_0$ 
		\State	2. Вычислить	
			\State \ \ \ \ \ \	$r_0 = X_0 - 2X_1+4X_2$
	  		\State \ \ \ \ \ \	$r_1 = X_0 - X_1+X_2$
			\State \ \ \ \ \ \	$r_2 = X_0$
	  		\State \ \ \ \ \ \	$r_3 = X_0 +X_1+X_2$
			\State \ \ \ \ \ \	$r_4 = X_0 + 2X_1+4X_2$
	  		\State \ \ \ \ \ \        $s_0 = Y_0 - 2Y_1+4Y_2$
	  		\State \ \ \ \ \ \	$s_1 = Y_0 - Y_1+Y_2$
			\State \ \ \ \ \ \	$s_2 = Y_0$
	  		\State \ \ \ \ \ \	$s_3 = Y_0 +Y_1+Y_2$
			\State \ \ \ \ \ \	$s_4 = Y_0 + 2Y_1+4Y_2$
		
		\State 3. Для i от 0 до 4 вычислить
			\State \ \ \ \ \ \  	$t_i = r_is_i$
		\State	4. Вычислить
			\State \ \ \ \ \ \	$Z_0 = t_2$
	  		\State \ \ \ \ \ \	$Z_1 = t_0/12 - 2t_1/3+2t_3/3-t_4/12$
			\State \ \ \ \ \ \	$Z_2 = -t_0/24 + 2t_1/3 -5t_2/4+2t_3/3-t_4/24$
	  		\State \ \ \ \ \ \	$Z_3 = -t_0/12 + t_1/6-t_3/6+t_4/12$
			\State \ \ \ \ \ \	$Z_4 = t_0/24 - t_1/6 +t_2/4-t_3/6+t_4/24$	
		\State	5. Восстановить Z по формуле
			\State \ \ \ \ \ \	$Z =Z_4B^4+  Z_3B^3+Z_2B^2 +Z_1B+Z_0$ 
		\State	6.
			\Return{Z}
	\end{algorithmic}

\end{algorithm}


\section{Машина 1}
Ноутбук:
\\$\triangleright$ Процессор: \textbf{Intel(R) Core(TM) i7-7700HQ CPU @2.80 GHz (3.80 Ghz Turbo Boost) (4 ядра и 8 потоков)}
\\$\triangleright$ Оперативная память: \textbf{8GB DDR4 2400 MHz }
\\$\triangleright$ OC: \textbf{Windows 10 Home, версия 1909 (сборка 18363.657) }

\section{Результаты}

%Изменить
\begin{tabular}{|c|c|c|c|}
\hline
X & Y & Sage Algorithm Time & My Algorithm Time\\
\hline
\hline
$123^{421897}$ & $987^{436112}$ & 5.15s & 0.04s\\
$12345^{421897}$ & $98765^{436112}$ & 13.70s & 0.09s\\
$1234567^{421897}$ & $9876543^{436112}$ & 27.10s & 0.14s\\
$123456789^{421897}$ & $987654321^{436112}$ & 44.50s & 0.19s\\
$1234567891011^{421897}$ & $9876543210123^{436112}$ & 90.00s & 0.40s\\
\hline
\end{tabular}

\section{Машина 2}
Стационарный пк:
\\$\triangleright$ Процессор: \textbf{Intel(R) Core(TM) i5-9600K CPU @3.70 GHz (4.60 Ghz Turbo Boost) (6 ядер и 6 потоков)}
\\$\triangleright$ Оперативная память: \textbf{16GB DDR4 3200 MHz }
\\$\triangleright$ OC: \textbf{Windows 10 Pro, версия 1903 (сборка 18362.239) }

\section{Результаты}
%изменить
\begin{tabular}{|c|c|c|c|}
\hline
X & Y  & Sage Algorithm Time & My Algorithm Time\\
\hline
\hline
$123^{421897}$ & $987^{436112}$ & 4.11s & 0.03s\\
$12345^{421897}$ & $98765^{436112}$ & 11.10s & 0.07s\\
$1234567^{421897}$ & $9876543^{436112}$ & 22.00s & 0.11s\\
$123456789^{421897}$ & $987654321^{436112}$ & 34.90s & 0.14s\\
$1234567891011^{421897}$ & $9876543210123^{436112}$ & 72.00s & 0.29s\\
\hline
\end{tabular}

%Изменить
\paragraph{Bibliography}:
\\ $[1]$ Львовский С.М. - Набор и верстка в системе LATEX. 5-е изд. 2014.
\\ $[2]$ Wikipedia: Wikipedia, The Free Encyclopedia [online], http://www.wikipedia.org, June 2020
\\ $[3]$ Zimmermann P., Casamayou A., Cohen N. et al. - Computational Mathematics with SageMath. 2018
\\ $[4]$ И.А. Панкратова Теоретико-числовые методы в криптографии
\\ $[5]$ Richard Crandall, Carl Pomerance - Prime Numbers. A Computational Perspective. Second Edition

\end{document}